% !TEX root = paper.tex

\documentclass[11pt,letterpaper]{article}

% Packages
\usepackage{packages}
\addbibresource{sources.bib}

% Document Settings
\geometry{margin=1in}
\setlength{\parskip}{1ex}
\setlength{\parindent}{0pt}

% =========================================
%             DOCUMENT
% =========================================

\begin{document}
\doublespacing % Double spacing throughout the document


% =========================
%      TITLE PAGE
% =========================

% Suppresses headers, footers, and page numbers on title page
\begin{titlepage}
    \begin{center}
    \vspace*{4cm}
    A statistical approach to predicting a tsunami’s wave height \\
    \vspace{1cm}
    Can the height of a tsunami's wave be predicted based on the distance from and 
    the magnitude of a preceding earthquake? \\
    \vspace{4cm}
    Word count: TBD
    \vfill
    \vspace{0.1cm}
    \end{center}
    \end{titlepage}

% =========================
%      DOCUMENT BODY
% =========================

\pagenumbering{roman}

\begin{center}
\pdfbookmark{\contentsname}{Contents}
\tableofcontents
\vspace{1in}

\end{center}


% TODO: Add hyper-references to links
% TODO: Improve listings' styling
% TODO: Add links to Contents entries

\newpage

\pagenumbering{arabic}

\section{Introduction}

\subsection{Background}

Naturally occurring disasters such as tsunamis are devastating for 
people and resources residing in risk areas. These tsunamis are caused 
by many different types of natural phenomena such as landslides, collapse of 
seamount, or more rarely, the impact of a meteorite. This paper however will focus 
on one of the more common cases of tsunami; one caused by the earthquakes. This 
is due to earthquakes causing around 72\% of earthquakes \cite{pacifictsunamimuseum}, and as the time between an 
earthquake event and its tsunami is relatively high compared to events such as landslides \cite{sue_nokes_walters}, 
leaving adequate time for prediction and correct preparation of the events properties, such as 
the wave height, as discussed in this paper. 

\subsection{Personal Engagement}

I have chosen to write about this subject as I have recently become interested by the 
statistics and its implications in real-life matters. This research has proved very 
interesting for me as it has abstracted the field of statistics for me. Furthermore, 
as I enjoy programming, the research I have done has helped me learn about the tools 
and technologies used by data analysts and other professionals of the matter. 


\section{}

\printbibliography[heading=bibintoc, title=Works Cited]



\end{document}