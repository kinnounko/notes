\section{Conclusion}

To conclude, even though Viète's method can seem more convoluted in practice, due to its geometric origins, 
it can be seen through the experiment conducted in this paper that this is clearly not the case 
in a computational context. His method is simplified to a point where its geometric origins are no 
longer findable from its computational implementation. And despite the assumption that could be 
made based on the types of operations needed for each method, Viète's requiring arguably more 
computationally challenging operations. These operations, such as square-root function or the product function 
used on high-precision floating point numbers, can be said to be more demanding than what could be said 
is simple summation, in the case of Madhava. This essay could be considered in defense of 
what is known as geometric algebra --- which can clearly produce results more capable than ones 
originating from an analytical method. 

\subsection{Evaluation of the experimental method}

It must be noted that the experimental procedure in this paper is far from perfect. Firstly, 
there are many variables that would be difficult to control. 

\textit{More info to be added}