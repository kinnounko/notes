\section{Introduction}


The value of $\pi$ has been researched for many years, although under different names,
and the amount of different approaches to reach the value is large. The value has been 
found through many processes, be it analytically, the most popular, geometrically, or through 
more obscure or convoluted methods, such as the possibility to approximate the value using 
physics from billiards. \cite{galperin_2003}

This paper seeks to examine the extent at which two historical methods of
approximation of the value $\pi$, namely the approaches suggested by the aforementioned 
mathematicians Madhava and Viète, differ in terms of computational efficiency and 
speed, and explain these differences. This paper does not however, 
suggest a method to use for computation but rather seeks to compare the efficiency of
a geometrically derived formula and a algebraically derived one. 