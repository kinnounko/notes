\section{Introduction}


The value of $\pi$ has been researched for many years, although under different names,
and the amount of different approaches to reach the value is large. The value has been 
found through many processes, be it analytically, the most popular, geometrically, or through 
more obscure or convoluted methods, such as the possibility to approximate the value using 
physics akin to those from a simple game of billiards. \cite{galperin_2003}

This paper seeks to examine the extent at which two historical methods of
approximation of the value $\pi$, namely the approaches suggested by the aforementioned 
mathematicians Madhava and Viète, differ in terms of computational speed and 
speed, and explain these differences. This paper does not however, 
suggest a method to use for computation but rather seeks to only compare 
the speed of a geometrically derived formula and a analytically derived one. 

This research could prove useful in the field of computer science, as 
there is always a demand for faster and more efficient programs in an
ever-changing society. Furthermore, a research based on computational speed of two different 
kinds approaches to the constant $\pi$ has not been done to date.  