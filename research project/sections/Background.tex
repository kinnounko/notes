\section{Theoretical approach}

\subsection{Focus on two methods}

For the sake of this paper, two different methods with similar 
convergence rates but different approaches have been chosen for 
comparison, the process for the original discovery of these methods
will be explained. The two mathematicians in question are Madhava 
of Sangamagramma of Medieval India and the French mathematician 
Viète. 



\subsubsection{Madhava's method}

Madhava, 


\subsubsection{Viète's method}

The French mathematician, François Viète, approached the value of 
$\pi$ from a geometric standpoint, finding the following formula:

$$\pi = 2 \frac{2}{\sqrt{2}} \frac{2}{\sqrt{2 + \sqrt{2}}} \frac{2}{\sqrt{2 + \sqrt{2 + \sqrt{2}}}} \dots$$

% $$\pi = \frac{2}{\prod_{k = 0}^{\infty} U_{k}}$$    symbolic
He was able to calculate $\pi$ to a place of 9 decimal points, in 
the year 1593 \cite{Kreminski}, using his method. His method consists 
of finding the area of a polygon of $n$ sides in a circle of constant 
radius. As the value of $n$ is increased, the area of the $n$-gon 
tends toward the area of a circle. The geometric origin of this formula 
can be found using simple right-angle trigonometry, by first finding 
the lengths $OH$ and subsequently $BD$ in Figure 1 (\ref{fig:image1}).

With the radius $R = OB$,

$OH = R \cos{\alpha}$ 

and

$BD = 2BH = 2 R \sin{\alpha}$

Since the equation for the area of a polygon is defined as $A = \frac{p \cdot a}{2}$, 
where $p$ is the perimeter of the polygon and $a$ is the apothem, in this case $BD \cdot n$ 
and $OH$ respectively. 

Let $A_{n}$ equal the area of the polygon with $n$ sides as such: \\
$A_{n} = \frac{OH \cdot BD \cdot n}{2}$ \\
$= \frac{R \cos{\alpha} \cdot 2 R \sin{\alpha} \cdot n}{2} = n R^2 \sin{\alpha} \cos{\alpha}$

And if n is multiplied by 2, the angle $\angle \alpha$ is divided by 2, 
and the new area becomes: 

$A_{2n} = 2n R^2 \sin{\frac{\alpha}{2}} \cos{\frac{\alpha}{2}}$

So it can be written that, by definition,

$\frac{A_{n}}{A_{2n}} = \frac{n R^2 \sin{\alpha} \cos{\alpha}}{2 n R^2 \sin{\frac{\alpha}{2}} \cos{\frac{\alpha}{2}}} = \frac{\sin{2 \alpha}}{2 \sin{\alpha}}$

Which through the trigonometric identity... // todo




\begin{figure}[!h]
\definecolor{qqwuqq}{rgb}{0,0.39215686274509803,0}
\definecolor{uuuuuu}{rgb}{0.26666666666666666,0.26666666666666666,0.26666666666666666}
\definecolor{xdxdff}{rgb}{0.49019607843137253,0.49019607843137253,1}
\definecolor{qqqqzz}{rgb}{0,0,0.6}
\begin{tikzpicture}[line cap=round,line join=round,>=triangle 45,x=1cm,y=1cm,scale=0.7]
\clip(-8.372520930232552,-8.50470697674418) rectangle (7.69259534883721,7.028316279069767);
\draw [shift={(-0.03355999206952884,0)},line width=1pt,color=qqwuqq,fill=qqwuqq,fill opacity=0.10000000149011612] (0,0) -- (0:1.3302325581395344) arc (0:44.77428433325932:1.3302325581395344) -- cycle;
\draw [line width=0.8pt,color=qqqqzz] (0,0) circle (6cm);
\draw [line width=1pt] (4.242640687119286,4.242640687119285)-- (6,0);
\draw [line width=1pt] (6,0)-- (4.242640687119286,-4.242640687119285);
\draw [line width=1pt] (4.242640687119286,-4.242640687119285)-- (4.242640687119286,4.242640687119285);
\draw [line width=1pt] (-6,0)-- (6,0);
\draw [line width=1pt] (-4.275936263984878,-4.209081736713964)-- (4.242640687119286,4.242640687119285);
\draw [line width=1pt] (-4.242640687119285,4.242640687119286)-- (4.242640687119286,-4.242640687119285);
\draw [line width=1pt] (-0.03355999206952884,0)-- (5.121320343559643,2.1213203435596424);
\draw [line width=1pt] (-4.242640687119285,4.242640687119286)-- (-4.275936263984878,-4.209081736713964);
\begin{scriptsize}
\draw [fill=xdxdff] (4.242640687119286,4.242640687119285) circle (2.5pt);
\draw[color=xdxdff] (4.4772465116279085,4.623665116279071) node {$B$};
\draw [fill=xdxdff] (-4.242640687119285,4.242640687119286) circle (2.5pt);
\draw[color=xdxdff] (-4.295311627906972,4.623665116279071) node {$A$};
\draw [fill=xdxdff] (-4.275936263984878,-4.209081736713964) circle (2.5pt);
\draw[color=xdxdff] (-4.336241860465112,-4.55) node {$C$};
\draw [fill=xdxdff] (6,0) circle (2.5pt);
\draw[color=xdxdff] (6.237246511627907,0.3873860465116302) node {$E$};
\draw [fill=xdxdff] (4.242640687119286,-4.242640687119285) circle (2.5pt);
\draw[color=xdxdff] (4.3772465116279085,-4.5) node {$D$};
\draw[color=black] (5.3595720930232575,2.3111069767441874) node {$H'$};
\draw [fill=xdxdff] (5.12,2.12) circle (2.5pt);
\draw[color=black] (3.947479069767444,-0.615404651162788) node {$H$};
\draw [fill=xdxdff] (4.242640687119286,0) circle (2.5pt);
\draw [fill=xdxdff] (0,0) circle (2.5pt);
\draw[color=uuuuuu] (0.00003720930232879,-0.3464558139534907) node {$O$};
\draw[color=qqwuqq] (1.5325953488372117,0.3873860465116302) node {$\alpha$};
\end{scriptsize}
\end{tikzpicture}
\label{fig:image1}
\caption{Circle with 1 segment from a $n$-gon with point $H$ and 2 
segments from an $2n$-gon, one of which on point $H'$, inscribed in 
a circle of radius $OB$, adapted from \cite{borisgourevitch2013}}
\end{figure}