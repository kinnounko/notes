% !TEX root = paper.tex

\documentclass[12pt,a4paper]{article}

% Packages
\usepackage{packages}
\addbibresource{sources.bib}

% Document Settings
\geometry{margin=1in}
\setlength{\parskip}{1ex}
\setlength{\parindent}{0pt}


\newcommand{\degre}{\ensuremath{^\circ}}

% =========================================
%             DOCUMENT
% =========================================

\begin{document}
\doublespacing % Double spacing throughout the document

% =========================
%      TITLE PAGE
% =========================

% Suppresses headers, footers, and page numbers on title page
\begin{titlepage}
    \begin{center}
    \vspace*{4cm}
        Comparing the speed of two methods of approximating the value of $\pi$ ---
        a computational approach\\
    \vspace{1cm}
    To what extent can a method of approximation of the value $\pi$ be computationally more efficient than another? \\
    \vspace{4cm}
    Word count: 
    \vfill
    \vspace{0.1cm}
    \end{center}
    \end{titlepage}

% =========================
%      DOCUMENT BODY
% =========================

\pagenumbering{roman}

\begin{center}
\pdfbookmark{\contentsname}{Contents}
\tableofcontents
\vspace{1in}

\end{center}


% TODO: Add hyper-references to links
% TODO: Improve listings' styling
% TODO: Add links to Contents entries

\newpage

\pagenumbering{arabic}

% Sections of essay

\section{Introduction}


The value of $\pi$ has been researched for many years, although under different names,
and the amount of different approaches to reach the value is large. The value has been 
found through many processes, be it analytically, the most popular, geometrically, or through 
more obscure or convoluted methods, such as the possibility to approximate the value using 
physics akin to those from a simple game of billiards. \cite{galperin_2003}

This paper seeks to examine the extent at which two historical methods of
approximation of the value $\pi$, namely the approaches suggested by the aforementioned 
mathematicians Madhava and Viète, differ in terms of computational speed and 
speed, and explain these differences. This paper does not however, 
suggest a method to use for computation but rather seeks to only compare 
the speed of a geometrically derived formula and a analytically derived one. 

This research could prove useful in the field of computer science, as 
there is always a demand for faster and more efficient programs in an
ever-changing society. Furthermore, a research based on computational speed of two different 
kinds approaches to the constant $\pi$ has not been done to date.  

\section{Theoretical approach}

\subsection{Focus on two methods}

For the sake of this paper, two different methods with similar 
convergence rates but different approaches have been chosen for 
comparison, the process for the original discovery of these methods
will be explained.


\subsubsection{Madhava's method}

Madhava, known by other famous mathematicians of his time, as having 
discovered 


\subsubsection{Viète's method}

\section{Computational approach}

\subsection{The variables}
The dependent variable of this experiment is the time taken $t$ by the program to approximate a given number $n$ of correct decimal value of the constant $\pi$. 

The value $n$ will be altered in order to avoid possible similar convergence rates at a small amount of decimal places, and multiple trials will be run to decrease margin of error. Other variables of the experiment will be controlled. For example, the experiment will be run on a same isolated system, a virtual machine, with a minimal amount of processes running simultaneously to avoid any possible variance in results. 


\subsection{Implementing in Python}

A Python application was programmed (see appendix) in order to run the two methods aforementioned, and manage the collection of data.

The program assigns the time before the execution of the method to a variable \verb|t1| with the \verb|time.time()| Python function. At each iteration of the method, the number of valid decimal places of the resultant approximation are counted. Once a specified decimal place is reached, determined by a programming \verb|if| statement, that can be mathematically expressed as:

$| \pi - \pi_{approx} | < 10^{-n}$, where $\pi_{approx}$ is the approximation by the method and $n$ defines the number of correct decimal places desired.

A new \verb|t2| time variable is assigned and the time taken, defined by $\verb|t2| - \verb|t1|$, is stored. This process is repeated for all specified decimal accuracies and for both methods. The times recorded were stored in a \verb|.csv| file for further analysis.

The \verb|mpmath| library was used for the floating point operations required for comparison between approximated values and the constant $\pi$ that wouldn't have been possible using standard Python libraries, due to floating-point limitations (computing term for decimal numbers). \cite{mpmath}

\section{Analysis of the results}
\newcolumntype{C}{>{\Centering\arraybackslash\hspace{0pt}}X}


\pgfplotstableread{
    X Y
    0.7601666451 5
    1.288158894 10
    2.050116062 15
    2.63982296 20
    3.091802597 25
    3.819538355 30
    4.442822933 35
    5.081979036 40
    5.706284046 45
    6.370962858 50
    7.020516396 55
    7.723451853 60
}\vietetable

\pgfplotstableread{
    X Y
    0.5919837952 5
    1.307063103 10
    1.902475357 15
    2.679748535 20
    3.212277889 25
    3.904948235 30
    4.547913074 35
    5.255579948 40
    5.837199688 45
    6.578524113 50
    7.270073891 55
    8.004565239 60

}\madhavatable


\pgfplotstableread{
    X Y
    10  5
    17  10
    27  15
    35  20
    42  25
    51  30
    59  35
    67  40
    76  45
    84  50
    92  55
    101 60
}\vietetableiter

\pgfplotstableread{
    X Y
    9 5
    20 10
    29 15
    41 20
    50 25
    61 30
    71 35
    82 40
    91 45
    102 50
    112 55
    123 60
}\madhavatableiter


\subsection{Presentation of the data}

\subsubsection{Tabular presentation}

The arithmetic mean of a decimal place was calculated for 100 trials and the data points
in the table shown below were found (rounded to 3 significant figures). The values for the decimal place
0 have been dropped as it would not be logical to include them. 

\begin{table}[h]
    \noindent% <-- important
    \setlength\tabcolsep{3pt} % default: 6pt
    \begin{tabularx}{\textwidth}{@{} l *{6}{C} @{}}
        \toprule
        Decimal place & Average time for approximation using Viete's method, in milliseconds (ms) & Average time for approximation using Madhava's method method, in milliseconds (ms) \\
        \midrule
        5             & 0.760                                                                     & 0.592                                                                              \\
        10            & 1.29                                                                      & 1.31                                                                               \\
        15            & 2.05                                                                      & 1.9                                                                                \\
        20            & 2.64                                                                      & 2.68                                                                               \\
        25            & 3.09                                                                      & 3.21                                                                               \\
        30            & 3.82                                                                      & 3.9                                                                                \\
        35            & 4.44                                                                      & 4.55                                                                               \\
        40            & 5.08                                                                      & 5.26                                                                               \\
        45            & 5.71                                                                      & 5.84                                                                               \\
        50            & 6.37                                                                      & 6.58                                                                               \\
        55            & 7.02                                                                      & 7.27                                                                               \\
        60            & 7.72                                                                      & 8.00
    \end{tabularx}
\end{table}

\begin{table}[h]
    \noindent% <-- important
    \setlength\tabcolsep{3pt} % default: 6pt
    \begin{tabularx}{\textwidth}{@{} l *{6}{C} @{}}
        \toprule
        Decimal place & Iterations needed, Viète's method (no unit) & Iterations needed, Madhava's method (no unit) \\
        \midrule
        5             & 10                                          & 9                                             \\
        10            & 17                                          & 20                                            \\
        15            & 27                                          & 29                                            \\
        20            & 35                                          & 41                                            \\
        25            & 42                                          & 50                                            \\
        30            & 51                                          & 61                                            \\
        35            & 59                                          & 71                                            \\
        40            & 67                                          & 82                                            \\
        45            & 76                                          & 91                                            \\
        50            & 84                                          & 102                                           \\
        55            & 92                                          & 112                                           \\
        60            & 101                                         & 123
    \end{tabularx}
\end{table}


The iterations needed for both methods were also measured, using the Python program
(see appendix), slightly modified to increment a variable \verb|i| and subsequently
return it, writing it in a similar manner to a \verb|.csv| file.



\subsubsection{Graphical presentation}

Shown below is a graph demonstrating the correlation between the decimal places approximated
and the amount of iterations needed for this specific decimal milestone. The blue markers and
line of best-fit respresent the results received from Viète's method while the red represent those
from Madhava's method. \\

\begin{tikzpicture}[scale=1.1]
    \begin{axis}[
            title={Comparing the iterations needed for approximating the value of $\pi$},
            xlabel={Iterations},
            ylabel={Decimal places},
            xmin=0, xmax=130,
            ymin=0, ymax=60,
            ytick={0,10,20,30,40,50,60},
            xtick={0,10,20,30,40,50,60,70,80,90,100,110,120,130},
            legend pos=south east,
            ymajorgrids=true,
            grid style=dashed,
            scale=1.2
        ]

        \addplot[
            only marks,
            color=blue,
            mark=*,
        ] table {\vietetableiter};
        \legend{Viète's method}

        \addplot [thin, blue] table[
                y={create col/linear regression={y=Y}}
            ] % compute a linear regression from the input table
            {\vietetableiter};
        \addlegendentry{best fit line}



        \addplot[
            only marks,
            color=red,
            mark=*,
        ] table {\madhavatableiter};
        \addlegendentry{Madhava's method}

        \addplot [thin, red] table[
                y={create col/linear regression={y=Y}}
            ] % compute a linear regression from the input table
            {\madhavatableiter};
        \addlegendentry{best fit line}

    \end{axis}
\end{tikzpicture}

To show the trend in the time data collected, see below the decimal places compared
to the time taken per method. The color coding here is relevant to the one aforementioned. \\

\begin{tikzpicture}[scale=1.1]
    \begin{axis}[
            title={Comparing the time taken for approximating the value of $\pi$},
            xlabel={Time taken [ms]},
            ylabel={Decimal places},
            xmin=0, xmax=10,
            ymin=0, ymax=60,
            ytick={0,10,20,30,40,50,60},
            xtick={0,2,4,6,8,10},
            legend pos=south east,
            ymajorgrids=true,
            grid style=dashed,
            scale=1.2
        ]

        \addplot[
            only marks,
            color=blue,
            mark=*,
        ] table {\vietetable};
        \legend{Viète's method}

        \addplot [thin, blue] table[
                y={create col/linear regression={y=Y}}
            ] % compute a linear regression from the input table
            {\vietetable};
        \addlegendentry{best fit line}



        \addplot[
            only marks,
            color=red,
            mark=*,
        ] table {\madhavatable};
        \addlegendentry{Madhava's method}

        \addplot [thin, red] table[
                y={create col/linear regression={y=Y}}
            ] % compute a linear regression from the input table
            {\madhavatable};
        \addlegendentry{best fit line}

    \end{axis}
\end{tikzpicture}




\subsection{Observations and analysis}

The results gathered from this experiment show linear relationship between the amount of decimal
places approximated and the time required for this approximation, as well as between the decimal places
and the amount of iterations needed. The two lines of best fit for the two methods show
similarities but also differences.

Where decimal places and iterations are compared, there is a clear difference between the two
methods: Madhava's method indeed takes more iterations to converge to a specific decimal of
the value $\pi$. However, in the second graph comparing time, it can be noted that Madhava's
method is marginally faster until the decimal 20, when the lines of the two methods stray apart. It could be
assumed that this occurrence is due to differences between the types of the two methods. Although it 
could be argued that this difference is only due to random error, the results in this graph 
are those collected from the arithmetic mean over 100 trials, and so it can be assessed that this is not the case. Madhava's
method is what is called an alternating series, due to its nature of alternating between values in order
to converge to a specific value, in this case $\pi$, while Viète's method is an infinite product. The
difference between these two methods is underlined in (Figure 3) \ref{fig:comparaison}. It can be
deduced that at lower decimal values Madhava's method is faster as Viète's method's approximation
accuracy firstly increases in what seems similar to a curve from an exponential function $f(x) = k^x$.
It can also be said that while Madhava's method alternates between values such that $m_{n} > \pi, m_{n+1} < \pi$,
Viète's method $\pi_v$ approximates pi in a manner such that $\pi_v < \pi$.

\begin{figure}[h]
    \includegraphics[width=\linewidth]{image.png}
    \caption{Comparison of historical methods of approximating the value of $\pi$. This image demonstrates
        their convergence rates. The line labelled Madhava in dark blue is the method used in this paper.
        From (Wikimedia Commons) \cite{infinite_series_comparaison}}
    \label{fig:comparaison}
\end{figure}


\section{Further research opportunities}

\subsection{}

\section{Conclusion}

To conclude, even though Viète's method can seem more convoluted in practice, due to its geometric origins, 
it can be seen through the experiment conducted in this paper that this is clearly not the case 
in a computational context. His method is simplified to a point where its geometric origins are no 
longer findable from its computational implementation. And despite the assumption that could be 
made based on the types of operations needed for each method, Viète's requiring arguably more 
computationally challenging operations. These operations, such as square-root function or the product function 
used on high-precision floating point numbers, can be said to be more demanding than what could be said 
is simple summation, in the case of Madhava. This essay could be considered in defense of 
what is known as geometric algebra --- which can clearly produce results more capable than ones 
originating from an analytical method. 

\subsection{Evaluation of the experimental method}

It must be noted that the experimental procedure in this paper is far from perfect. Firstly, 
there are many variables that would be difficult to control. 

\textit{More info to be added}

\printbibliography[heading=bibintoc, title=Works Cited]

\appendix
\section{Appendix}
\label{app}
\subsection{Python program}
\label{app:scripts}
This application was run on Python version 3.9.1, on a virtual machine 
running the Debian operating system under QEMU/KVM on a Intel i5-2500 
processor. Used the \verb|mpmath| library for better floating-point 
precision \cite{mpmath}. 
\lstinputlisting[language=Python]{madhavaviete.py}



\end{document}